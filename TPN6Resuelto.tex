\documentclass[11pt]{examdesign}
\usepackage[spanish]{babel}
\OneKey
\usepackage[utf8]{inputenc}
\usepackage[T1]{fontenc}
\usepackage{amsmath}
\usepackage{pifont}
\usepackage{graphicx}
\usepackage{float}
\usepackage{amscd}
\usepackage{amsfonts}
\usepackage{amssymb}
\usepackage{mathtools}
\usepackage{amsthm}
\usepackage[all]{xy}
\usepackage{enumitem}
\usepackage{multicol}
\usepackage{multirow}
\usepackage{verbatim}
\usepackage[colorlinks=true,
breaklinks=true,
linkcolor=blue,
urlcolor=red,
bookmarksopen=true]{hyperref}
\usepackage[pdftex,dvipsnames]{xcolor}
\definecolor{aqua}{rgb}{0.0, 1.0, 1.0}
\definecolor{caribbeangreen}{rgb}{0.0, 0.8, 0.6}
\definecolor{tealgreen}{rgb}{0.0, 0.51, 0.5}
\definecolor{upforestgreen}{rgb}{0.0, 0.27, 0.13}
\definecolor{napiergreen}{rgb}{0.16, 0.5, 0.0}
\definecolor{capri}{rgb}{0.0, 0.75, 1.0}
\definecolor{calpolypomonagreen}{rgb}{0.12, 0.3, 0.17}
\definecolor{azure(colorwheel)}{rgb}{0.0, 0.5, 1.0}
\definecolor{dukeblue}{rgb}{0.0, 0.0, 0.61}
\definecolor{bole}{rgb}{0.47, 0.27, 0.23}
\definecolor{gris}{gray}{0.975}
%----------------------------------------------------------------------------------------------
% Si desea utilizar \@@line para definir su propio encabezado de examen o palabras del encabezado, 
% asegúrese de usar \makeatletter y \makeatother en los lugares apropiados, de lo contrario 
% podría obtener errores.
%-----------------------------------------------------------------------------------------------
\theoremstyle{plain}
\newtheorem{theorem}{Theorem}[section]
\newtheorem{thm}[theorem]{Teorema}
\newtheorem{cor}[theorem]{Corolario}
\newtheorem{lem}[theorem]{Lema}
\newtheorem{pro}[theorem]{Proposición}
\newtheorem{axs}[theorem]{Axiomas}
\newtheorem{axi}[theorem]{Axioma}
\theoremstyle{definition}
\newtheorem{exas}[theorem]{Ejemplos}
\newtheorem{exa}[theorem]{Ejemplo}
\newtheorem{defi}[theorem]{Definición}
\theoremstyle{remark}
\newtheorem{rmk}[theorem]{Observación}
\newtheorem{step}{Step}
\newtheorem{xca}[theorem]{Ejercicio}
\newtheorem{prob}[theorem]{Pregunta}
\newtheorem{rmks}[theorem]{Observaciones}
\newtheorem*{proofmt}{Prueba del Teorema Principal}
\usepackage[centerlast,small,sc]{caption}
\setlength{\captionmargin}{20pt}
\newcommand{\axref}[1]{Axioma~\ref{#1}}
\newcommand{\defref}[1]{\textbf{Definición}~\ref{#1}}
\newcommand{\coref}[1]{\textbf{Corolario}~\ref{#1}}
\newcommand{\thref}[1]{\textbf{Teorema}~\ref{#1}}
\newcommand{\lref}[1]{\textbf{Lema}~\ref{#1}}
\newcommand{\exaref}[1]{Ejemplo~\ref{#1}}
\newcommand{\xcaref}[1]{Ejercicio~\ref{#1}}
\newcommand{\rmkref}[1]{Observación~\ref{#1}}
\newcommand{\pref}[1]{\textbf{Proposición}~\ref{#1}}
\newcommand{\fref}[1]{Figura~\ref{#1}}
\newcommand{\tref}[1]{Tabla~\ref{#1}}
\newcommand{\cref}[1]{\textbf{Capítulo}~\ref{#1}}
\newcommand{\sref}[1]{\textbf{Sección}~\ref{#1}}
\newcommand{\aref}[1]{Apéndice~\ref{#1}}
\newcommand{\eref}[1]{Ecuación~\eqref{#1}}
\newcommand{\dref}[1]{Diagrama~\eqref{#1}}
\usepackage{makeidx}
\usepackage{tikz,tkz-tab}%
\usetikzlibrary{matrix,arrows,positioning,shadows,shadings,backgrounds,
	calc, shapes, tikzmark}
\usepackage{tcolorbox, empheq}%
\tcbuselibrary{skins,breakable,listings,theorems}

\tcbset{opteqC/.style={skin=beamer,colback=red!1!white}}
\newcommand{\celda}[2]{
	                  \begin{minipage}{#1mm}
	                      \centering
	                      \vspace{2mm}
		                      #2
		                  \vspace{2mm}
                      \end{minipage}
                     }
\makeatletter
%----------------------------------------------------------------------------------------------%
\begin{examtop}
	\@@line{\parbox{3in}{\classdata \\[0.5cm]
			\textcolor{upforestgreen}{\textbf{\underline{T.P.N$^\circ$}}~\fbox{\textsc{6}}-\textbf{Continuación:}}~\examtype}
		%                  ^^^^^^
		\hfill
		\parbox{3in}{\normalsize \namedata}}
	\bigskip
\end{examtop}
%----------------------------------------------------------------------------------------------%
\def\namedata{\textcolor{upforestgreen}{\textbf{Estudiante}}:\hrulefill \\[\namedata@vspace]
	          \textcolor{upforestgreen}{\textbf{Curso y División}}: 2do año, IV-VI \\[\namedata@vspace]
	          \textcolor{upforestgreen}{\textbf{Profesor}}: Ferreira, Juan David \\[\namedata@vspace]
	          \textcolor{upforestgreen}{\textbf{Fecha de Entrega}}:\hrulefill 
          }
 % manual page 11
%----------------------------------------------------------------------------------------------% 
\begin{keytop}%
	\@@line{\hfill \Huge\texttt{\textcolor{upforestgreen}{Respuestas 
		Trabajo Práctico N$^\circ$~\fbox{\textsc{6}}}} \hfill}
	\bigskip
\end{keytop}%
%----------------------------------------------------------------------------------------------%
\makeatother
\examname{\textcolor{upforestgreen}{\underline{\textbf{Fracciones}}}}

\SectionPrefix{\textcolor{upforestgreen}{\textbf{Sección \arabic{sectionindex}}.} \space}
\Fullpages
\ContinuousNumbering
\DefineAnswerWrapper{}{}
\NumberOfVersions{1}
\class{{\textcolor{upforestgreen}{\large\textbf{E.P.E.S. Nro 51 ``J. G. A.''}}\\[0.5cm]
		\textcolor{upforestgreen}{{\large \textbf{Matemática}}}}}
%----------------------------------------------------------------------------------------------%
\begin{document}
	%-------------------------------       SHORT ANSWER        ------------------------%
    \begin{shortanswer}[title={\textit{Receta que le dijo la Abuela Pocha a Ana y
    			Juan.}},
    	rearrange=no,resetcounter=no]
    	\vspace*{-0.2cm}
    	\begin{figure}[!h]
    		\begin{minipage}[b]{0.3\textwidth}% 30% de la página
    			\begin{center}% Figuras: ver capítulo 5
    				\includegraphics{nena_chipa.png}
    				%\caption{Poliedros}
    			\end{center}
    		\end{minipage}
    		\hfill
    		\begin{minipage}[b]{0.65\textwidth} % 65% de la página
    			\textbf{\textit{\textcolor{upforestgreen}{Ingredientes}}}:
    			\begin{itemize}
    				\item 300 g de cebolla picada,
    				
    				\item  1 kg (1000 g) de choclo
    				cocido desgranado,
    				
    				\item 200 g de grasa de cerdo,
    				
    				\item 300 g de queso,
    				
    				\item  5 huevos batidos,
    				
    				\item 1 taza de leche,
    				
    				\item sal y pimienta a gusto
    			\end{itemize}
    		\end{minipage}
    	\end{figure}
        \vspace{0.2cm}
        \textbf{\textit{\textcolor{upforestgreen}{Preparación}}}:
        \vspace{0.2cm}
        Saltear la cebolla picada en la grasa, agregar la leche y dejar cocinar $10$ minutos más, retirar del fuego y agregar el queso, los huevos batidos y el choclo. Mezclar bien y colocar en una asadera enmantecada, llevar al horno caliente y sacar cuando está dorado por encima. (\textcolor{dukeblue}{Texto de la receta de \textit{Chipa Guazú} extraído del libro formoseño “Gastronomía Formoseña". SANTANDER, Jorge M}).
        %--------------------------------------------------------------------------------------%
    	\begin{question}
    		Completá el siguiente cuadro con las calorías que aporta cada ingrediente, en la receta de la abuela Pocha
    		\begin{center}
    			\begin{tabular}{|c|c|c|c|}
    				\hline 
    				\multirow{2}{*}{Ingredientes que poseen Ana y Juan} &Fracción que&
    				Fracción que&Fracción que
    				\\
    				                          &  van a usar   &les falta & les sobra
    				\\\hline  
    				Cebolla frita = $500g$    &               &          &    
    				\\\hline
    				Choclo cocido = $100g$    &               &          &    
    				\\\hline
    				Grasa de cerdo = $100g$   &               &          &    
    				\\\hline
    				Queso = $750g$            &               &          &    
    				\\\hline
    				Huevo cocido = $12$ unidad &              &          &    
    				\\\hline
    				Leche =$3$ tazas           &              &          &    
    				\\\hline
    			\end{tabular}
    		\end{center}
    		\begin{answer}
    			\textbf{Repuesta}: En el primer ingrediente tienen \textbf{Cebolla frita = $500g$} y la receta dice que necesitan \textbf{300 g de cebolla picada}. Entonces, usarian $300g$ de los $500g$, en decir, en fracción representa $300/500=3/5$ del total de cebolla que disponen. Les sobra $2/5$.
    			\begin{center}
    				\begin{tabular}{|c|c|c|c|}
    					\hline 
    					\multirow{2}{*}{Ingredientes que poseen Ana y Juan} &Fracción que&
    					Fracción que&Fracción que
    					\\
    					&  van a usar   &les falta   & les sobra
    					\\\hline  
    					Cebolla frita = $500g$    & $300/500=3/5$ & $-$        & $2/5$   
    					\\\hline
    				\end{tabular}
    			\end{center}
    			
    			En el segundo ingrediente tienen \textbf{Choclo cocido = $100g$} pero se necesita \textbf{1 kg (1000 g) de choclo cocido desgranado}. Entonces, usarian $100g$ no les alcana para toda la receta, yq que se requieren $1000g$, en decir, en les falta $900g$ del total Choclo cocido. Ahora veamos con fraaciones: tienen 100 gramos de 1000g requeridos, eso se representa como $100/1000=1/10$ y les faltan $900g$, es decir $900/1000=9/10$.
    			
    			\begin{center}
    				\begin{tabular}{|c|c|c|c|}
    					\hline 
    					\multirow{2}{*}{Ingredientes que poseen Ana y Juan} &Fracción que&
    					Fracción que&Fracción que
    					\\
    					&  van a usar   &les falta   & les sobra
    					\\\hline  
    					Cebolla frita = $500g$    & $300/500=3/5$ & c        & $2/5$   
    					\\\hline
    					\textbf{Choclo cocido} = $100g$    & $100/100$     & $900/1000$ & $-$    
    					\\\hline
    				\end{tabular}
    			\end{center}
    		
    		   En el tercer ingrediente tienen \textbf{Grasa de cerdo = $100g$} pero se necesita \textbf{200 g de grasa de cerdo}. Entonces, usarian $100g$ no les alcana para toda la receta, ya que se requieren $200g$, en decir, en les falta $100g$ del total de grasa de cerdo. Ahora veamos con fraaciones: tienen 100 gramos de 200g requeridos, eso se representa como $100/200=1/2$ y les faltan $100g$, es decir $100/200=1/2$.
    		   
    		   \begin{center}
    		   	\begin{tabular}{|c|c|c|c|}
    		   		\hline 
    		   		\multirow{2}{*}{Ingredientes que poseen Ana y Juan} &Fracción que&
    		   		Fracción que&Fracción que
    		   		\\
    		   		&  van a usar   &les falta   & les sobra
    		   		\\\hline  
    		   		Cebolla frita = $500g$    & $300/500=3/5$ & $-$        & $2/5$   
    		   		\\\hline
    		   		Choclo cocido = $100g$    & $100/100$     & $900/1000$ & $-$    
    		   		\\\hline
    		   		\textbf{Grasa de cerdo} = $100g$          &   $100$    & $1/2$ &$-$          
    		   		\\\hline
    		   	\end{tabular}
    		   \end{center}
    		\end{answer}
    	\end{question}
        
        %--------------------------------------------------------------------------------------%
    \end{shortanswer}
    %------------------------------------------------------------------------------------------%
    \begin{shortanswer}[title={\textit{Operaciones con fracciones.}},
    	rearrange=no,resetcounter=yes]
    	%--------------------------------------------------------------------------------------%
    	\begin{question}
    		Determine el valor numérico de cada expresión, expresando la respuesta en su forma más simple.
    		\begin{enumerate}
    			\item $6\times\left(\frac{4}{3}-\frac{13}{6}\right)+\frac{1}{2}=$
    			
    			\item $3\times\left(\frac{3}{4}-\frac{5}{16}\right)+\left(\frac{5}{2}+\frac{7}{2}\right)=$
    			
    			\item $\frac{1}{3}\times\left(\frac{6}{3}-\frac{4}{5}\right)-\frac{4}{2}\times\left(\frac{5}{3}+\frac{2}{6}\right)=$
    			
    			\item $\left(\frac{9}{4}+\frac{3}{2}\times{5}{6}\right)+\frac{1}{2}=$
    			
    			\item $3\left(\frac{4}{12}+\frac{7}{6}\right)-\frac{9}{3}\times\frac{16}{8}=$
    			
    			\item $15\left(\frac{6}{5}-\frac{13}{15}\right)+\frac{3}{11}\left(\frac{7}{3}+\frac{15}{3}\right)=$
    		\end{enumerate}
    		
    		\begin{answer}
    			\textbf{Repuesta}: Completar sus respuestas. No se olviden de eso, ya que solo resuelvo algunos ejercicios a modo de ejemplo.
    			
    			\begin{enumerate}
    				\item $6\times\left(\frac{4}{3}-\frac{13}{6}\right)+\frac{1}{2}=$
    				      
    				      $6\times\left(\frac{4\times\textbf{2}}{3\times\textbf{2}}-\frac{13}{6}\right)+\frac{1}{2}=$
    				      
    				      $6\times\left(\frac{8}{\textbf{6}}-\frac{13}{\textbf{6}}\right)+\frac{1}{2}=$
    				      
    				      $6\times\left(\frac{8-13}{\textbf{6}}\right)+\frac{1}{2}=$
    				      
    				      $6\times\left(\frac{-5}{\textbf{6}}\right)+\frac{1}{2}=$
    				      
    				      $\left(\frac{-30}{\textbf{6}}\right)+\frac{1\times\textbf{3}}{2\times\textbf{3}}=$
    				      
    				      $\left(\frac{-30}{\textbf{6}}\right)+\frac{3}{\textbf{6}}=$
    				      
    				      $\left(\frac{-30+3}{\textbf{6}}\right)=\textcolor{red}{\frac{-27}{\textbf{6}}}$
    				\item $3\times\left(\frac{3}{4}-\frac{5}{16}\right)+\left(\frac{5}{2}+\frac{7}{2}\right)=$
    				
    				$3\times\left(\frac{3\times\textbf{4}}{4\times\textbf{4}}-\frac{5}{16}\right)+\left(\frac{5+7}{2}\right)=$
    				
    				$3\times\left(\frac{12}{\textbf{16}}-\frac{5}{\textbf{16}}\right)+\left(\frac{5+7}{2}\right)=$
    				
    				$3\times\left(\frac{12-5}{\textbf{16}}\right)+\frac{12}{2}=$
    				
    				$3\times\frac{7}{\textbf{16}}+\frac{12}{2}=$
    				
    				$\frac{21}{\textbf{16}}+\frac{12\times\textbf{8}}{2\times\textbf{8}}=$
    				
    				$\frac{21}{\textbf{16}}+\frac{96}{\textbf{16}}=$
    				
    				$\frac{21+96}{\textbf{16}}=\textcolor{red}{\frac{117}{16}} $
    				
    				\item $\frac{1}{3}\times\left(\frac{6}{3}-\frac{4}{5}\right)-\frac{4}{2}\times\left(\frac{5}{3}+\frac{2}{6}\right)=$
    				
    				\item $\left(\frac{9}{4}+\frac{3}{2}\times{5}{6}\right)+\frac{1}{2}=$
    				
    				\item $3\left(\frac{4}{12}+\frac{7}{6}\right)-\frac{9}{3}\times\frac{16}{8}=$
    				
    				\item $15\left(\frac{6}{5}-\frac{13}{15}\right)+\frac{3}{11}\left(\frac{7}{3}+\frac{15}{3}\right)=$
    			\end{enumerate}
    		\end{answer}
    	\end{question}
    	
    \end{shortanswer}
    %------------------------------------------------------------------------------------------%
    \begin{shortanswer}[title={\textit{Más problemas con fracciones...}},
    	rearrange=no,resetcounter=yes]
    	%--------------------------------------------------------------------------------------%
    	\begin{question}
    		%--------------------------------------------------------------------------------------%
    		Calcular cuántas manzanas y fresas tenemos en la tienda si en total hay $240$ frutas, sabiendo:
    		%--------------------------------------------------------------------------------------%
    		\begin{enumerate}
    			\item  Una sexta parte son manzanas,
    			
    			\item  Una tercera parte son fresas,	
    		\end{enumerate}
    		
    		\begin{answer}
    			\textbf{Repuesta}: El total es de $240$ frutas y nos dice que una sexta parte son manzanas. Esto se quiere decir $1/6$ de $240$ son manzanas:
    			
    			$$240/6=40 \mbox{ son manzanas. }$$ 
    			
    			El total es de $240$ frutas y nos dice que una tercera parte son fresas. Esto se quiere decir $1/3$ de $240$ son fresas:
    			
    			$$240/3=80 \mbox{ son fresas. }$$
    		\end{answer}
    	\end{question}
    
    	\begin{question}
    		¿Hay algún otro tipo de fruta en la tienda? ¿Cuántas?
    		
    		\begin{answer}
    			\textbf{Repuesta}: Si e un total de $240$ frutas tenemos que $40$ son manzanas y $80$ son fresas entonces $240-40-80=120$ seran de algún otro tipo de fruta.
    		\end{answer}
    	\end{question}
    
        \begin{question}
        	 ¿Qué fracción del total representa?
        	
        	\begin{answer}
        		\textbf{Repuesta}:  Ya sabemos que $120$ son frutas que son de algún otro tipo y se representa como $120/240=1/2$.
        	\end{answer}
        \end{question}
    \end{shortanswer}
\end{document}