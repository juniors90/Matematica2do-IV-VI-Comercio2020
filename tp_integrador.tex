\documentclass[10pt]{examdesign}
\usepackage[spanish]{babel}
\OneKey
\usepackage[utf8]{inputenc}
\usepackage[T1]{fontenc}
\usepackage{amsmath}
\usepackage{pifont}
%-----------------------------------------------------------------------------------------------
%\usepackage{gfsartemisia-euler}
\usepackage{graphicx}
\usepackage{float}
\usepackage{amscd}
\usepackage{amsfonts}
\usepackage{amssymb}
\usepackage{mathtools}
\usepackage{amsthm}
\usepackage[all]{xy}
\usepackage{enumitem}
\usepackage{multicol}
\usepackage{verbatim}
\usepackage[colorlinks=true,
linkcolor=blue,
urlcolor=red,
bookmarksopen=true]{hyperref}
\usepackage[pdftex,dvipsnames]{xcolor}
\definecolor{aqua}{rgb}{0.0, 1.0, 1.0}
\definecolor{caribbeangreen}{rgb}{0.0, 0.8, 0.6}
\definecolor{tealgreen}{rgb}{0.0, 0.51, 0.5}
\definecolor{upforestgreen}{rgb}{0.0, 0.27, 0.13}
\definecolor{napiergreen}{rgb}{0.16, 0.5, 0.0}
\definecolor{capri}{rgb}{0.0, 0.75, 1.0}
\definecolor{calpolypomonagreen}{rgb}{0.12, 0.3, 0.17}
\definecolor{azure(colorwheel)}{rgb}{0.0, 0.5, 1.0}
\definecolor{dukeblue}{rgb}{0.0, 0.0, 0.61}
\definecolor{bole}{rgb}{0.47, 0.27, 0.23}
\definecolor{gris}{gray}{0.975}
%-------------------------------------------------------------------------------------------------%
% Si desea utilizar \@@line para definir su propio encabezado de examen o palabras del encabezado, 
% asegúrese de usar \makeatletter y \makeatother en los lugares apropiados, de lo contrario 
% podría obtener errores.
%-------------------------------------------------------------------------------------------------%
\makeatletter
% manual page 10
\begin{examtop}
	\@@line{\parbox{3in}{\classdata \\[0.5cm]
			\textcolor{upforestgreen}{\textbf{\underline{T.P.N$^\circ$}~\fbox{\textsc{7}}~:}} \examtype}
		%                  ^^^^^^
		\hfill
		\parbox{3in}{\normalsize \namedata}}
	\bigskip
\end{examtop}
% manual page 10
\def\namedata{\textcolor{upforestgreen}{\underline{\textbf{Estudiante}}}:\hrulefill \\[\namedata@vspace]
	\textcolor{upforestgreen}{\underline{\textbf{Curso y División}}}: 2$^\circ$ Año \\[\namedata@vspace]
	\textcolor{upforestgreen}{\underline{\textbf{División}}}: IV-VI
	\\[\namedata@vspace]
	\textcolor{upforestgreen}{\underline{\textbf{Profesor}}}: Ferreira, Juan David \\[\namedata@vspace]
	\textcolor{upforestgreen}{\underline{\textbf{Fecha}}}:\hrulefill
}
% manual page 11        
\begin{keytop}%
	\@@line{\hfill \Huge\texttt{\textcolor{upforestgreen}{Respuestas 
				Trabajo Práctico N$^\circ$~\fbox{\textsc{6}}}} \hfill}
	\bigskip
\end{keytop}%
\makeatother


\examname{\textcolor{upforestgreen}{\underline{\textbf{Actividedes de Integración}}}}
%\SectionPrefix{Sección \arabic{sectionindex}. \space}
%\SectionPrefix{Sección \Alph{sectionindex}. \space}
%\SectionPrefix{Punto \Alph{sectionindex}. \space}
%\SectionPrefix{Punto \arabic{sectionindex}. \space}
%\SectionPrefix{Ejercicio \Alph{sectionindex}. \space}
%\SectionPrefix{\textcolor{upforestgreen}{\textbf{Ejercicio \arabic{sectionindex}}.} \space}
\SectionPrefix{\textcolor{upforestgreen}{\textbf{Parte \Alph{sectionindex}}.} \space}
%\SectionPrefix{\textcolor{upforestgreen}{\textbf{Parte \arabic{sectionindex}}.} \space}

%-----------------------------------------------------------------------------------------------%
% Este macro toma un argumento, quiere decir que dentro de las llaves vamos a colocar un estilo de
% letra para nuestro titulo. hay que saber acerca de fuentes y estilos de letras.
%-----------------------------------------------------------------------------------------------%

%\SectionFont{\Large\rmfamily}
%\SectionFont{\Large\sffamily}
%\SectionFont{\Large\ttfamily}
%\SectionFont{\Large\mdseries}
%\SectionFont{\Large\bfseries}
%\SectionFont{\Large\upshape}
%\SectionFont{\Large\slshape}
%\SectionFont{\Large\itshape}
%\SectionFont{\Large\scshape}

%--------------------- Nos da los margenes-Funciona igual que el paquete geometry -------------%
\Fullpages

\ContinuousNumbering
%-------------------------------      RESPUESTAS CORTAS     -----------------------------------%
%\ShortKey

\DefineAnswerWrapper{}{}
%-----------------------------   NUMERO DE VERSIONES DEL EXAMEN   -----------------------------%
\NumberOfVersions{1}




%%%%%%%%%%%%%%%%%%%%%%%%%%%%%%%%%%%%%%%%%%%%%%%%%%%%%%%%%%%%%%%%%%%%%%%%%%%%%%%%%%%%%%%%%%%%%%%%%
% Este macro toma un argumento, quiere decir que dentro de las llaves vamos a colocar un estilo de letra para nuestro titulo. hay que saber acerca de fuentes y estilos de letras.
%%%%%%%%%%%%%%%%%%%%%%%%%%%%%%%%%%%%%%%%%%%%%%%%%%%%%%%%%%%%%%%%%%%%%%%%%%%%%%%%%%%%%%%%%%%%%%%%%

\class{{\textcolor{upforestgreen}{\Large\textbf{E.P.E.S. Nro 51 ``J. G. A.''}}\\[0.5cm]
		\textcolor{upforestgreen}{{\Large \textbf{Matemática}}}}
	}

\begin{document}
	
	
    %-------------------------------             SHORT ANSWER N°1        ------------------------%
    \begin{shortanswer}[title={\textbf{Resolver las siguientes situaciones mostrando en tu carpeta el razonamiento:}},
    	rearrange=no,resetcounter=no]
    	
        
        \begin{question}
            Marcia consultó el saldo de su cuenta el $15$ de Mayo y tenía $-\$3200 $. Si realizó un depósito de $\$1750$. ¿Qué saldo quedó en su cuenta luego de ese depósito?.
        	\begin{answer}
        		\textbf{Repuesta}: este ejercicio así planteado no tiene solución.
        	\end{answer}
        \end{question}
    
        \begin{question}
        	El 16 de mayo, Marcia realizó una extracción de $\$1200$. ¿Cuál es el saldo luego de esa extracción?.
        	\begin{answer}
        		\textbf{Repuesta}: este ejercicio así planteado no tiene solución.
        	\end{answer}
        \end{question} 
        
        \begin{question}
        	Sebastián está en el décimo piso de un edificio, le dicen que tiene que bajar a un depósito que se encuentra en piso $-4$ (cuarto subsuelo). ¿Cuántos pisos debe bajar?.
        	\begin{answer}
        		\textbf{Repuesta}: este ejercicio así planteado no tiene solución.
        	\end{answer}
        \end{question}
        
        \begin{question}
        	Una sabio murió en el año $23$ antes de Cristo y tenía $78$ años en ese momento. ¿En qué año nació?.
        	\begin{answer}
        		\textbf{Repuesta}:.
        	\end{answer}
        \end{question}        
    \end{shortanswer}
    
    %------------------------     COMPLETAR ESPACIOS EN BLANCOS N°2     -----------------------%
    \begin{fillin}[title={\textbf{Calculos de sumas y restas...}}, resetcounter=no, rearrange=no]
    	\begin{question}
    		Realizar las siguientes sumas y restas de acuerdo a la definición
    		\begin{itemize}
    			\item [a.] $15-(-32)=$\blank{woodchuck}
    			\item [b.] $-8-(-19)=$\blank{woodchuck}
    			\item [c.] $50-(-56)=$\blank{woodchuck}
    			\item [d.] $-5-30=$\blank{woodchuck}
    			\item [e.] $-15-(-6)=$\blank{woodchuck}
    			\item [f.] $95-(-100)=$\blank{woodchuck}
    			\item [g.] $-25-(-35)=$\blank{woodchuck}
    			\item [h.] $-15-20=$ \blank{woodchuck}
    		\end{itemize}
    	\end{question}	        	
    \end{fillin}

    %-------------------------------             SHORT ANSWER        ------------------------%
    \begin{shortanswer}[title={\textbf{Repasamos algunos conceptos...}},
    	rearrange=no,resetcounter=no]
    	
    	
    	\begin{question}
    		Completar el siguiente cuadro con números enteros:
    		
    		\begin{tabular}{|c|c|c|c|c|c|c|}
    			\hline
    			\textbf{A}&\textbf{B}&\textbf{Opuesto de A}&\textbf{Valor Absoluto de B}&\textbf{Siguiente de A}& \textbf{Anterior de A}&\textbf{A-B}
    			\\\hline
    			  $-13$          & $33$  & $13$  &  $33$ & $-12$ & $-14$ & $-13-33=-46$ 
    			\\\hline
    			  $32$           & $-54$ &       &       &       &  &
    			\\\hline
    			                 & $21$  &       &       & $-24$ &  &
    			\\\hline
    			  $41$           &       &       & $6$   &       &  &
    			\\\hline
    			$-90$            & $62$  &       &       &       &  &
    			\\\hline
    		\end{tabular}
    	
    		\begin{answer}
    			\textbf{Repuesta}:.
    		\end{answer}
    	\end{question}        
    \end{shortanswer}
     
    %------------------------     FILLIN N°4     -----------------------%
  
    %-----------------------------------------------------------------------------------------------%
    
\end{document}