\documentclass[11pt]{examdesign}
\usepackage[spanish]{babel}
\OneKey
\usepackage[utf8]{inputenc}
\usepackage[T1]{fontenc}
\usepackage{amsmath}
\usepackage{pifont}
%-----------------------------------------------------------------------------------------------
%\usepackage{gfsartemisia-euler}
\usepackage{graphicx}
\usepackage{float}
\usepackage{amscd}
\usepackage{amsfonts}
\usepackage{amssymb}
\usepackage{mathtools}
\usepackage{amsthm}
\usepackage[all]{xy}
\usepackage{enumitem}
\usepackage{multicol}
\usepackage{verbatim}
\usepackage[colorlinks=true,
linkcolor=blue,
urlcolor=red,
bookmarksopen=true]{hyperref}
\usepackage[pdftex,dvipsnames]{xcolor}
\definecolor{aqua}{rgb}{0.0, 1.0, 1.0}
\definecolor{caribbeangreen}{rgb}{0.0, 0.8, 0.6}
\definecolor{tealgreen}{rgb}{0.0, 0.51, 0.5}
\definecolor{upforestgreen}{rgb}{0.0, 0.27, 0.13}
\definecolor{napiergreen}{rgb}{0.16, 0.5, 0.0}
\definecolor{capri}{rgb}{0.0, 0.75, 1.0}
\definecolor{calpolypomonagreen}{rgb}{0.12, 0.3, 0.17}
\definecolor{azure(colorwheel)}{rgb}{0.0, 0.5, 1.0}
\definecolor{dukeblue}{rgb}{0.0, 0.0, 0.61}
\definecolor{bole}{rgb}{0.47, 0.27, 0.23}
\definecolor{gris}{gray}{0.975}
%----------------------------------------------------------------------------------------------
% Si desea utilizar \@@line para definir su propio encabezado de examen o palabras del encabezado, 
% asegúrese de usar \makeatletter y \makeatother en los lugares apropiados, de lo contrario 
% podría obtener errores.
%-----------------------------------------------------------------------------------------------
\makeatletter
% manual page 10
\begin{examtop}
	\@@line{\parbox{3in}{\classdata \\[0.5cm]
			\textcolor{upforestgreen}{\textbf{\underline{T.P.N$^\circ$}}~\fbox{\textsc{5}}} \examtype}
		%                  ^^^^^^
		\hfill
		\parbox{3in}{\normalsize \namedata}}
	\bigskip
\end{examtop}
% manual page 10
\def\namedata{\textcolor{upforestgreen}{\underline{\textbf{Estudiante}}}:\hrulefill \\[\namedata@vspace]
	          \textcolor{upforestgreen}{\underline{\textbf{Curso y División}}}: 2$^\circ$ Año \\[\namedata@vspace]
	          \textcolor{upforestgreen}{\underline{\textbf{División}}}: IV-VI
	          \\[\namedata@vspace]
	          \textcolor{upforestgreen}{\underline{\textbf{Profesor}}}: Ferreira, Juan David \\[\namedata@vspace]
	          \textcolor{upforestgreen}{\underline{\textbf{Fecha}}}: 16-06-2020 al 26-06-2020
          }
 % manual page 11        
\begin{keytop}%
	\@@line{\hfill \Huge\texttt{\textcolor{upforestgreen}{Respuestas 
		Trabajo Práctico N$^\circ$~\fbox{\textsc{5}}}} \hfill}
	\bigskip
\end{keytop}%
\makeatother


\examname{\textcolor{upforestgreen}{\underline{\textbf{Ecuación Exponencial}}}}
%\SectionPrefix{Sección \arabic{sectionindex}. \space}
%\SectionPrefix{Sección \Alph{sectionindex}. \space}
%\SectionPrefix{Punto \Alph{sectionindex}. \space}
%\SectionPrefix{Punto \arabic{sectionindex}. \space}
%\SectionPrefix{Ejercicio \Alph{sectionindex}. \space}
\SectionPrefix{\textcolor{upforestgreen}{\textbf{Ejercicio \arabic{sectionindex}}.} \space}
%%%%%%%%%%%%%%%%%%%%%%%%%%%%%%%%%%%%%%%%%%%%%%%%%%%%%%%%%%%%%%%%%%%%%%%%%%%%%%%%%%%%%%%%%%%%%%%%%

% Este macro toma un argumento, quiere decir que dentro de las llaves vamos a colocar un estilo de
% letra para nuestro titulo. hay que saber acerca de fuentes y estilos de letras.

%%%%%%%%%%%%%%%%%%%%%%%%%%%%%%%%%%%%%%%%%%%%%%%%%%%%%%%%%%%%%%%%%%%%%%%%%%%%%%%%%%%%%%%%%%%%%%%%%

%\SectionFont{\Large\rmfamily}
%\SectionFont{\Large\sffamily}
%\SectionFont{\Large\ttfamily}
%\SectionFont{\Large\mdseries}
%\SectionFont{\Large\bfseries}
%\SectionFont{\Large\upshape}
%\SectionFont{\Large\slshape}
%\SectionFont{\Large\itshape}
%\SectionFont{\Large\scshape}

%%%%%%%%%%%%%%%%%%%%%%%%      Nos da los margenes                        %%%%%%%%%%%%%%%%%%%%%%%%%%%
%%%%%%%%%%%%%%%%%%%%%%%%%     Funciona igual que el paquete geometry     %%%%%%%%%%%%%%%%%%%%%%%%%%%
\Fullpages

\ContinuousNumbering
%%%%%%%%%%%%%%%%%%%%%%%%%%%%%%%%%      RESPUESTAS CORTAS     %%%%%%%%%%%%%%%%%%%%%%%%%%%%%%%%%%%%%%%%
%\ShortKey

\DefineAnswerWrapper{}{}
%%%%%%%%%%%%%%%%%%     NUMERO DE VERSIONES DEL EXAMEN FILA A, FILA B, FILA C, ETC     %%%%%%%%%%%%%%%
\NumberOfVersions{1}




%%%%%%%%%%%%%%%%%%%%%%%%%%%%%%%%%%%%%%%%%%%%%%%%%%%%%%%%%%%%%%%%%%%%%%%%%%%%%%%%%%%%%%%%%%%%%%%%%
% Este macro toma un argumento, quiere decir que dentro de las llaves vamos a colocar un estilo de letra para nuestro titulo. hay que saber acerca de fuentes y estilos de letras.
%%%%%%%%%%%%%%%%%%%%%%%%%%%%%%%%%%%%%%%%%%%%%%%%%%%%%%%%%%%%%%%%%%%%%%%%%%%%%%%%%%%%%%%%%%%%%%%%%

\class{{\textcolor{upforestgreen}{\Large\textbf{E.P.E.S. Nro 51 ``J. G. A.''}}\\[0.5cm]
		\textcolor{upforestgreen}{{\Large \textbf{Matemática}}}}}

\begin{document}
	
	
    %-------------------------------             SHORT ANSWER        ------------------------%
    \begin{fillin}[title={Completar el siguiente cuadro siguiendo el ejemplo del primer renglón:}]
    	\begin{multicols}{2}
    		% 30% de la página
    		\begin{tabular}{|c|c|c|c|}
    			\hline
    			{\scriptsize Fracción}      &{\scriptsize Expresión}&{\scriptsize Nombre}	 &{\scriptsize Expresión}
    			\\
    			&{\scriptsize Decimal}  &            &{\scriptsize Porcentual}
    			\\\hline
    			$\frac{3}{4}$ & $0.75$  &{\scriptsize Tres cuartos}&$75\%$
    			\\\hline
    			$\frac{1}{4}$ &         &           &             
    			\\\hline
    			& $0.5$   &           &
    			\\\hline
    			&         &{\scriptsize Un quinto}&
    			\\\hline
    			$1$           & $0.8$   &           &
    			\\\hline
    			$\frac{3}{5}$ &         &           &
    			\\\hline
    			&         &{\scriptsize siete medios}&
    			\\\hline
    		\end{tabular}
    		\columnbreak
    		\begin{tabular}{|c|c|c|c|}
    			\hline
    			{\scriptsize Fracción}      &{\scriptsize Fracción}     &{\scriptsize Fracción}	 &{\scriptsize Fracción}
    			\\
    			&{\scriptsize Simplificada} &{\scriptsize Amplificada} &{\scriptsize Ireducible}
    			\\\hline
    			$\frac{3}{4}$ & $0.75$  &{\scriptsize Tres cuartos}&$75\%$
    			\\\hline
    			$\frac{1}{4}$ &         &           &             
    			\\\hline
    			& $0.5$   &           &
    			\\\hline
    			&         &{\scriptsize Un quinto}&
    			\\\hline
    			$1$           & $0.8$   &           &
    			\\\hline
    			$\frac{3}{5}$ & \blank{tres quintos}        &           &
    			\\\hline
    			&         &{\scriptsize siete medios}&
    			\\\hline
    		\end{tabular}
    	\end{multicols}
        \begin{question}
    	    How much \blank{wood} would a \blank{woodchuck} chuck, if a \blank{woodchuck}
    	    would \blank{chuck}, wood?
        \end{question}
    \end{n{fillin}

    
    %-----------------------------------------------------------------------------------------------%
    \begin{endmatter}
    	\vspace*{1.5in}
    	\centerline{\Large \textcolor{upforestgreen}{Material de consulta:}}
    	\bigskip
    	De la función exponencial podemos armar una tabla de valores (ya conocida). \footnote{Video de consulta \url{https://www.youtube.com/watch?v=Fl1PvjOh9Us}}
    	\begin{multicols}{2}
    		% 30% de la página
    		   
    			\begin{center}
    				\begin{tabular}{|l|c|}
    					\hline
    					$x$ & $f(x)=2^x$       \\ \hline
    					$-3$&$2^{-3}=\left(\frac{1}{2}\right)^3=\left(\frac{1}{8}\right)$\\\hline
    					$-2$&$2^{-2}=\left(\frac{1}{2}\right)^2=\left(\frac{1}{4}\right)$ \\\hline
    					$-1$& $2^{-1}=\left(\frac{1}{2}\right)^1=\left(\frac{1}{2}\right)$\\\hline
    					$0$ & $2^0=1$                   \\\hline
    					$1$          & $2^1=2$           \\\hline
    					$2$          & $2^2=2\cdot2=4$       \\\hline
    					$3$          & $2^3=2\cdot2\cdot2=8$     \\\hline
    					$\vdots$     & $\vdots$          \\\hline
    					$\textbf{m}$ & $2^m$    \\\hline
    				    $\textbf{n}$ & $2^n$    \\\hline
    				    $\textbf{t}$ & $2^t$    \\\hline
    				\end{tabular}
    		\end{center}
    		\columnbreak
    		\begin{center}
    			 \begin{tabular}{|l|c|}
    			 	\hline
    			 	$x$ & $f(x)=2^x$       \\ \hline
    			 	$-3$&$2^{-3}=\left(\frac{1}{2}\right)^3=\left(\frac{1}{8}\right)$\\\hline
    			 	$-2$&$2^{-2}=\left(\frac{1}{2}\right)^2=\left(\frac{1}{4}\right)$ \\\hline
    			 	$-1$& $2^{-1}=\left(\frac{1}{2}\right)^1=\left(\frac{1}{2}\right)$\\\hline
    			 	$0$ & $2^0=1$                   \\\hline
    			 	$1$          & $2^1=2$           \\\hline
    			 	$2$          & $2^2=2\cdot2=4$       \\\hline
    			 	$3$          & $2^3=2\cdot2\cdot2=8$     \\\hline
    			 	$\vdots$     & $\vdots$          \\\hline
    			 	$\textbf{m}$ & $2^m$    \\\hline
    			 	$\textbf{n}$ & $2^n$    \\\hline
    			 	$\textbf{t}$ & $2^t$    \\\hline
    			 \end{tabular}
    		 \end{center}
    	\end{multicols}     
    	
    	
    \end{endmatter}
    
\end{document}